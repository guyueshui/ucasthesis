\chapter{总结与展望}\label{chap:summ}

\section{全文总结}
本文主要研究了两种基于生成对抗网络的分类模型:C-InfoGAN和InfoCatGAN。在科技飞速发展的现代,大量未处理的原始数据正等待人们去研究,去挖掘,而数据预处理和数据标注是非常耗费精力的,所以如何无监督处理和挖掘这些数据中的价值已成为当下亟待解决的问题。本文研究的基于GAN的分类模型可以无监督或半监督地对数据分类,同时还能生成指定类别的数据样本,具有相当程度的研究意义。现将全文内容总结如下:

第一章作为绪论,介绍了当下人工智能、深度学习领域的快速发展,奠定了本文的研究背景。接着介绍了生成对抗网络及其在各种场景下的应用,引出分类问题。最后介绍了GAN在分类领域的研究现状,提出本文的主要工作并给出文章结构安排。

第二章介绍了一些预备知识,包括生成式模型和判别式模型的区别,生成对抗网络基本思想和数学模型,详细介绍了两篇前人的工作并以此阐述了使用生成对抗网络做分类的理论推导和实际方法。

第三章介绍了本文的主要工作:C-InfoGAN和InfoCatGAN。详细分析了第二章提出的两种模型的特点并有针对性地提出改进方案,将互信息的物理意义与损失函数结合,分别给出了无监督和半监督条件下的损失函数推导、具体训练方法和模型结构图,得到的损失函数具有一定的可解释性。

第四章给出了模型在MNIST和FashionMNIST数据集上的评估结果。首先给出了实验设定和实现细节以并详细介绍了评估方法,接着给出了评价指标。然后分别给出了两个模型在两个数据集上的结果,最后给出了收敛速度分析。

本文的研究兼顾理论与实践,既有问题的形式化定义和损失函数的理论推导,也给出了详实的实验细节和实验步骤。本文将信息论中的互信息与生成对抗网络联系起来,从而得到了生成效果和分类性能的提升,是对信息论与深度学习交叉结合的一次探索。从本文的结果来看,信息论在深度学习领域具有值得探索的价值。

\section{未来展望}

限于本文作者的能力,时间和精力,本文所做的研究工作虽有一定贡献,但仍存在一些值得继续探索和深入挖掘的方向,具体如下:
\begin{itemize}
  \item 两个模型的分类准确率均较低,如何进一步提高准确率仍需要研究;
  \item 本文仅在CatGAN的基础上添加了互信息正则,就能够显著提升生成效果,可见互信息在其中起了重要作用,其中相关原理尚待探索;
  \item 在InfoCatGAN模型中,互信息正则项仅体现在隐变量和生成数据之间,而没有对真实数据和判别器做相应的设计,对于判别器一方,原则上也需要设计相应的正则项;
  \item 本文在实验中均假设训练数据集是类别均匀的,对于非均衡数据集的研究仍是需要解决的问题。
\end{itemize}