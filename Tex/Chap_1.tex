\chapter{绪论}\label{chap:introduction}
近年来,人工智能(Artificial Intelligence,AI)领域的蓬勃发展时刻影响着人们的
生活方式。从现金结账到刷脸支付,从出租车到无人驾驶,从人工咨询到导航机器人,
从超市购物到无人零售,从洗衣拖地到智能家居,这些都是过去几年科技的发展给
人类生活带来的影响。科技的背后往往都是技术的发展,这其中尤为重要的一个就是
人工智能技术。事实上人``人工智能''的概念早在1956年就被提出,当时几个计算机
科学家梦想着用当时刚刚出现的计算机来构造复杂的、拥有与人类智慧同样本质的机器。
之后,人工智能一直萦绕在人们的脑海,并在科研实验室中缓慢孵化。后来的几十年间,
人工智能发展的不温不火,它一方面代表了人类文明的谣言未来,一方面又被当成技术
狂想被仍在角落。2012年以后,得益于数据量的上涨、计算力的提升以及深度学习的出现,
人工智能才开始爆发。据领英大数据显示,截至2017年一季度全球人工智能领域专业技术
人才数量超过190万,其中美国人工智能领域专业技术人才超过85万,高居榜首。中国
人工智能领域专业技术人才超过5万人,且人才缺口达到500多万。印度、英国、加拿大
和澳大利亚分列2--5位。
与此同时,人工智能的研究领域也在不断扩大,图~\ref{fig:ai}展示了人工智能研究的
一些分支,包括专家系统、机器学习、推荐系统等。其中,机器学习是一种实现人工
智能的方法。它使用算法来解析数据,从中学习,然后对真实世界中的事件做出决策和
预测。不同于传统的为解决特定任务而设计的程序,机器学习是使用大量数据来训练,
通过各种方法从数据中学习如何完成任务。
\begin{figure}[htbp]
  \centering
  \includegraphics[width=0.9\textwidth]{Img/ai-content.pdf}
  \bicaption{人工智能研究分支}{Research branches of AI}
  \label{fig:ai}
\end{figure}

机器学习直接来源于早期的人工智能领域,传统的算法包括:决策树、聚类、朴素贝叶斯、支持向量机、Expectation Maximization (EM)、AdaBoost等。从学习方法上又可分为
监督学习(如分类)、无监督学习(如聚类)、半监督学习、集成学习、深度学习和
强化学习。其中深度学习是最近几年来的热门研究领域。从上面的划分来看,深度学习
并不是一种独立的学习方法,而是一种实现机器学习的技术。只是近年来深度学习领域
的创新方法层出不穷,使得越来越多的学者都将其看作为一种独立的学习方法。
总而言之,越来越多的学者都踊跃加入深度学习研究的行列,为深度学习活跃发展及
应用做出极大的贡献。

\section{研究背景}
近些年,机器学习在人们生活的方方面面发挥着极大的作用:语音识别、交通预测、
视频监控、垃圾邮件过滤、智能客服、商品推荐等等。然而,机器学习算法需要从
大量数据集中提取有用的特征,进而做出决策。但是目前并没有一个通用的特征提取
方法适用于不同场景。因此学者们提出了一个新的方法,称为表征学习
(Representation Learning),它能够在分类和检测任务中自动提取有用的
特征\cite{bengio2013representation}。
深度学习\cite{lecun2015deep}也属于一种表征学习方法,它可以提取出高维度、
更抽象的数据表征。然而,这些方法都属于监督学习,需要数据集具有多样化的特征,
并且每个数据样本都需要被标注。通常获取这些标注好的数据集或者去标注数据集的难度
很大。所以,越来越多的学者将重心放在无监督学习上。在无监督学习中,
生成式模型(见\ref{sec:gm-dm}节)是相对有效的方法,早期的生成式模型
(如Restricted Boltzmann Machines (RBM)\cite{smolensky1986information},
Deep Belief Networks (DBN)\cite{hinton2006fast}, Deep Boltzmann Machines
\cite{salakhutdinov2009deep}
)
通常基于马尔可夫链、极大似然和近似推断等工具。然而这些方法的计算复杂度都很高,
并且无法保证良好的泛化性能。近年来,生成对抗网络
(Generative Adversarial Networks, GAN)\cite{goodfellow2014generative}作为
一种新的生成式模型受到了广泛的关注。该模型包含两个网络:生成器和判别器。
在对抗训练的过程中,生成器会学习到数据的概率分布,同时生成逼真的数据去迷惑
判别器,而判别器的目标是从虚假数据中区分出真实数据。两个网络之间通过相互对抗,
最终达到一个平衡点。值得一提的是,GAN不需要使用复杂的近似推断或马尔可夫链,
并且在计算机视觉,自然语言处理及其他众多领域都有较好的生成效果。

% 生成对抗网络最大的贡献在于它提出了一种对抗的思想。
% 这种思想已经成功应用于众多
% 领域如计算机视觉、自然语言处理等。前些年的AlphaGo\cite{silver2016mastering}打败
% 了人类顶尖围棋手的事件让AI引起了广泛关注,而AlphaGo的中间阶段就运用了对抗训练的
% 思想。对抗样本\cite{kurakin2016adversarial,athalye2018obfuscated,
% goodfellow2014explaining,kos2018adversarial}也利用了对抗的思想。
% 所谓对抗样本是指那些与真实数据
% 差别很小却被判为异常的,或者与真实数据相差很大却被判为真实类别的样本。

% GAN变体以及应用非常丰富。如前所述,GAN在不需要知道真实数据分布,
% 不需要做额外假设的情况下,可以通过随机噪声生成逼真的数据样本。这使得GAN在计算
% 机视觉和图像处理领域的应用大获成功。
GAN在图像超分辨率方面有很好的效果。SRGAN\cite{ledig2017photo}是第一个实现图像
超分辨率的生成对抗网络模型,\citet{wang2018esrgan}通过对SRGAN的三个主要组成
部分进行改进得到了ESRGAN,比如,它利用\citet{jolicoeur2018relativistic}中提出
的方法令判别器输入相对的真实距离而非绝对距离。
基于CycleGAN\cite{zhu2017unpaired},\citet{yuan2018unsupervised}提出
Cycle-in-Cycle GAN用于无监督的图像超分辨率。\citet{ding2019tgan}提出TGAN,
通过探索张量结构来生成大型高质量图像。

GAN在图像生成方面也有很多应用。\citet{tran2017disentangled}提出DR-GAN用于对姿态鲁棒的人脸识别。\citet{huang2017beyond}提出了一种通过同时感知局部细节和整体结构来生成逼真的正面视图的GAN(TP-GAN)。\citet{ma2017pose}提出了一种使用姿势指导生成人体图像的生成网络(PG$^2$),该网络基于一个新的姿势和该人体的图像来合成任意姿势的该人体图像。
% \citet{cao2018learning}提出了一种基于GAN的高保真姿态不变性模型,用于生成高分辨率正面人脸。
虽然大多数论文使用GAN来合成二维图像\cite{bao2017cvae,dong2017semantic},但\citet{wu2016learning}使用GAN和三维卷积生成了三维样本,他们生成了新颖的物体包括汽车,椅子,沙发和桌子。\citet{im2016generating}使用循环对抗网络生成图像。\citet{yang2017lr}提出了用于图像生成的分层循环生成对抗网络(LR-GAN)。

GAN也可以用于目标检测。考虑学习一个对变形和遮挡鲁棒的物体检测器,一种方法是使用具有变形和遮挡的数据集来训练。但是一些变形和遮挡非常罕见,它们在现实中可能不会发生。此时可以用GAN来生成这样的数据以参与训练。\citet{wang2017fast}使用GAN生成具有变形和遮挡的实例,对手的目的是生成难以被对象检测器分类的样本。通过将segmentor和GAN相结合,Segan\cite{ehsani2018segan}能够检测到图像中其他物体遮挡的物体。为了解决小物体检测问题,\citet{li2017perceptual}]提出感知生成对抗网络模型;\citet{bai2018sod}提出了一种端到端的多任务GAN(MTGAN)。

此外,GAN在分类问题中也有所应用,这也是本中的研究重点。关于使用GAN做分类,

- CatGAN
- SGAN 
- tripleGAN
- triangleGAN
- ETGAN
- InfoGAN

% from semi-supervised learning by entropy minimization
In the probabilistic framework, semi-supervised learning can be modeled as a missing data problem, which can be addressed by generative models such as mixture models thanks to the EM algorithm and extensions thereof [6].Generative models apply to the joint density of patterns and class (X, Y ). They have appealing features, but they also have major drawbacks. Their estimation is much more demanding than discriminative models, since the model of P(X, Y ) is exhaustive, hence necessarily more complex than the model of P(Y |X). More parameters are to be estimated, resulting in more uncertainty in the estimation process. The generative model being more precise, it is also more likely to be misspecified. Finally, the fitness measure is not discriminative, so that better models are not necessarily better predictors of class labels. These difficulties have lead to proposals aiming at processing unlabeled data in the framework of supervised classification [1, 5, 11].  Here, we propose an estimation principle applicable to any probabilistic classifier, aiming at making the most of unlabeled data when they are beneficial, while providing a control on their contribution to provide robustness to the learning scheme.

在概率框架中,可以将半监督学习建模为丢失的数据问题,由于EM算法及其扩展[6],可以通过生成模型(例如混合模型)来解决该问题。生成模型适用于模式的联合密度和类(X,Y)。它们具有吸引人的功能,但也有许多缺点。他们的估计比判别模型要严格得多,因为P(X,Y)模型是穷举性的,因此必定比P(Y | X)模型更复杂。要估计更多的参数,导致估计过程中更多的不确定性。生成模型更精确,也更可能被错误指定。最后,适应度度量没有区别,因此,更好的模型不一定是类别标签的更好预测指标。这些困难导致提出了旨在在监督分类框架内处理未标记数据的提议[1、5、11]。在这里,我们提出一种适用于任何概率分类器的估计原理,旨在在有益的情况下充分利用未标记的数据,同时控制其贡献,从而为学习方案提供鲁棒性

\section{研究现状}
\section{本文贡献}
\section{本文结构}
