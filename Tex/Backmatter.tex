%---------------------------------------------------------------------------%
%->> Backmatter
%---------------------------------------------------------------------------%
\chapter{作者简历及攻读学位期间发表的学术论文与研究成果}


\section*{作者简历}

胡兵兵,安徽马鞍山人,中国科学院上海微系统与信息技术研究所硕士研究生。

\section*{已发表(或正式接受)的学术论文:}

{
\setlist[enumerate]{}% restore default behavior
\begin{enumerate}[nosep]
    \item \textbf{Hu, B.}; Wang, K.; Ma, Y.; Wu, Y. On the Capacity Regions of Degraded Relay Broadcast Channels with and without Feedback. Entropy 2020, 22, 784.
    % \item \textbf{胡兵兵},唐华,吴幼龙. 基于互信息约束的生成对抗网络分类模型研究,中国科学院大学学报,2020. (尚在评审)
\end{enumerate}
}

% \section*{申请或已获得的专利:}
% 
% (无专利时此项不必列出)
% 
% \section*{参加的研究项目及获奖情况:}
% 
% 可以随意添加新的条目或是结构。

\chapter[致谢]{致\quad 谢}\chaptermark{致\quad 谢}% syntax: \chapter[目录]{标题}\chaptermark{页眉}
\thispagestyle{noheaderstyle}% 如果需要移除当前页的页眉
%\pagestyle{noheaderstyle}% 如果需要移除整章的页眉

不知不觉又过了三个年头,依稀记得三年前我刚来到学校的时候还对硕士生涯充满着未知,不知道这三年又会是什么样的生活。如今回过头来发现,越是学习,越是发现自己不懂的东西很多。犹记得在学习课程时期,经常熬夜到凌晨写作业,做实验的情景;犹记得每天读论文,每周开组会,上台做报告情景;犹记得某些问题弄不懂,抓耳挠腮和同学讨论,请教老师的情景。这些经历都是我宝贵的回忆,以后也会在我前行的道路上给予我动力。

首先我要感谢我的导师吴幼龙。作为一名年轻的教授,他具有严谨的学术态度、专业的知识素养、灵活的思维方式、以及浓烈的科研热情。更难能可贵的是,他平易近人,丝毫没有老师的架子,与学生相处亦师亦友。在求学途中,我的导师总是孜孜不倦、不厌其烦地教导我,经常与我讨论问题到很晚,无论是学习还是生活上,都给予了我莫大的关怀。在完成毕业论文期间,我更是频繁地向他请教问题,讨论实验方案,汇报论文进展,他也非常耐心地为我解答,和我讨论,帮我修改论文。我在此向他表示由衷的感谢!我还要感谢我的同窗们:汪科、马莹莹,以及同组的学弟学妹陈家慧、唐华,他们陪我一起度过了求学的时光,我们经常一起讨论学习和生活上的各种问题。我还要感谢我的学长学姐们:黄曦、高欣、边思梦,他们在我刚来学校的那段时间,给了我很多的帮助,让我能快速地融入学校,适应研究生阶段的学习模式。我也要感谢我的父母,是他们给了我求学的机会,给我提供了生活的基础。此外,我还要感谢我的女朋友尚玉静,她一直默默支持我,鼓励我,陪伴我。都说每个成功男人的背后都有一个伟大的女人,我可能不够成功,她可能不够伟大,但我觉得我们能走到一起这件事就是我最大的成功。

最后,感谢审阅论文的老师们,感谢百忙之中出席答辩的老师们!我在此向所有关心、支持和帮过我的人表示由衷的感谢!

\cleardoublepage[plain]% 让文档总是结束于偶数页,可根据需要设定页眉页脚样式,如 [noheaderstyle]
%---------------------------------------------------------------------------%
